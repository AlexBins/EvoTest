\begin{frame}{Steuersignalerzeugung}
Im Anschluss an die Geometrieermittlung folgt die Berechnung der Steuersignale.

Es existieren zwei primitive:
\begin{itemize}
\item Gerade
\item Kreise
\end{itemize}
Beide sind in Klassen gekapselt, die Schnittstelle
\[ctrl\_signal = calcCtrlSignal(velocity,\,axis\_length)\]
dient dann der Erzeugung des Kontrollvektors:
\[[velocity\,steering\_angle\,duration]\]
\end{frame}

\subsection{Geraden}
\begin{frame}{Steuersignale -- Geraden}
Für geraden ist die Steuersignalerzeugung trivial:
\begin{itemize}
\item Der Lenkwinkel ist Null
\item Die Dauer des Steuersignals ist abhängig von der zurückzulegenden Entfernung $x$ und der Geschwindigkeit $v$ und kann aus $v=\frac{s}{t}$ abgeleitet werden: $t=\frac{s}{v}$
\end{itemize}
Die Geschwindigkeit entspricht der gewünschten Geschwindigkeit.
\end{frame}

\subsection{Kreise}
\begin{frame}{Steuersignalerzeugung -- Kreise}
Kreise sind komplexer:
\begin{itemize}
\item Der Lenkwinkel hängt von der Krümmung des Kreises und damit von dessen Radius sowie dem Achsabstand ab:  $atan(achsabstand/radius)$
\item Die Dauer hängt von der Wunschgeschwindigkeit und der Länge des Kreissegments ab: \[\frac{Winkelsumme\cdot Radius}{Geschwindigkeit}\]
\end{itemize}
Die Geschwindigkeit entspricht der gewünschten Geschwindigkeit.
\end{frame}