\documentclass[12pt,a4paper]{scrartcl}
\usepackage[utf8]{inputenc}
\usepackage[german]{babel}
\usepackage[T1]{fontenc}

\title{Evolutionärer Algorithmus -- EvoTest}
\author{Yaroslax, Alexander, Manuel}

\begin{document}
\section{Genetischer Algorithmus}
Der genetische Algorithmus dient zur iterativen Optimierung der Testfälle. Er erzeugt eine Anzahl Cromosome, die über mehrere Stufen durch diverse Operationen verbessert werden.

\subsection{Allgemeines}
% Alex, evntl UML, umsetzung, etc

\subsection{Chromsome}
% Parameterisierung (Prozent der Range)
% Manuel

\subsection{Austauschbare Operatoren}
% Faktory-Pattern, Einbindung in gen GA
% Manuel, Alex

\section{Operatoren}
Die Operatoren, die im genetischen Algorithmus Anwendung finden, sind modular tauschbar. Sie müssen lediglich die Schnittstellendefinition einhalten.
\subsection{Initialisierung}
% Yaroslav

\subsection{Mutation}
% Manuel

\subsection{Rekombination}
% Wer hat die Rekombination gemacht?

\subsection{Fitnes}
% Manuel, evntl Alex nach Verbesserung?
% Wir sollten den Verbesserungsprozess beschreiben, evntl ein paar Effekte der Primärversion

\subsection{Selektion}
% Alex?
\end{document}